%%%%%%%%%%%%%%%%%%%%%%%%%%%%%%%%%%%%%%%%%%%%%%%%%%%%%%%%%%%%%%%%%%%%%%%%%%%%%%%%
% ficha.tex
%
% Modelo de arquivo para uso com a classe faesaTeX2, para a formatação de
% trabalhos acadêmicos na FAESA Centro Universitário (https://www.faesa.br).
%
% Para maiores informações, visite:
%    https://github.com/abrantesasf/faesatex2
%
% Este arquivo cria ficha catalográfica de sua monografia, automaticamente, de
% acordo com os dados que foram fornecidos pelo autor no arquivo da monografia,
% no formato padronizado seguido pela maioria das universidades e faculdades,
% por exemplo:
%    https://fichacatalografica.ufam.edu.br/ficha/create
%    https://www.rsirius.uerj.br/novo/index.php/ficha2
%    https://sistemas.ufmt.br/mfc/
%    http://fichacatalografica.ufc.br/
%    https://portal.biblioteca.ufabc.edu.br/servicos/ficha-catalografica
%    https://portal.biblioteca.ufabc.edu.br/ficha_catalografica/
%    https://www2.ufjf.br/biblioteca/ficha-catalografica/
%    http://biblioteca.unip.br/FichaCatalografica/BIBFichaCatalograficaWEB.aspx
%    https://sabi.ufrgs.br/servicos/publicoBC/ficha.php
%    https://www.tabelacutter.com/
%    https://cuttersonline.com/app/pages/home
%
% Você deve utilizar este modelo até a aprovação final do trabalho e, após isso,
% se a biblioteca da FAESA lhe fornecer uma ficha catalográfica definitiva,
% deve substituir o conteúdo deste arquivo por uma imagem da ficha final. Siga
% as instruções abaixo.
%%%%%%%%%%%%%%%%%%%%%%%%%%%%%%%%%%%%%%%%%%%%%%%%%%%%%%%%%%%%%%%%%%%%%%%%%%%%%%%%


%%%%%%%%%%%%%%%%%%%%%%%%%%%%%%%%%%%%%%%%%%%%%%%%%%%%%%%%%%%%%%%%%%%%%%%%%%%%%%%%
% Ficha catalográfica provisória:
% É gerada automaticamente, não ALTERE NADA AQUI!
\begin{fichacatalografica}
\vspace*{\fill}
\begin{center}
Dados Internacionais de Catalogação na Publicação (CIP)
\vspace{0.05cm}

\fbox{
\begin{minipage}[c][8cm]{1.5cm}
\vspace{0.4cm}
\imprimircutter
\vspace*{\fill}
\end{minipage}
\begin{minipage}[c][8cm]{12.5cm}
\vspace{0.4cm}
\imprimirautorcutter

\hspace{0.5cm} \imprimirtitulo\  / \imprimirautor. --- \imprimirdata.

\vspace{0.4cm}
\hspace{0.5cm} \thelastpage\ págs.\ :\ il.\ color.\ ;\ 30 cm.

\vspace{0.4cm}
\hspace{0.5cm} \imprimirorientadorRotulo~\imprimirorientador

\ifthenelse{\isundefined{\coorientador}}{}{
\hspace{0.5cm} \imprimircoorientadorRotulo~\imprimircoorientador
}

\hspace{0.5cm} \imprimirtipotrabalho\ (\imprimircursofaesa) ---
\imprimirnomefaesa, \imprimirunidadefaesa, \imprimirlocal, \imprimirdata.

\vspace{0.4cm}
\hspace{0.5cm} \imprimirassunto
\vspace*{\fill}
\end{minipage}
}
\vspace{-0.1cm}

Gerado automaticamente com os dados fornecidos pelo(a) autor(a).
\end{center}
\end{fichacatalografica}


%%%%%%%%%%%%%%%%%%%%%%%%%%%%%%%%%%%%%%%%%%%%%%%%%%%%%%%%%%%%%%%%%%%%%%%%%%%%%%%%
% Ficha catalográfica provisória:
% Depois que seu trabalho estiver aprovado de forma definitiva, se a biblioteca
% da FAESA lhe forneceu uma ficha catalográfica oficial e definitiva, salve
% essa filha em formato PDF. Depois, comente ou apague o código acima, da
% ficha catalográfica provisória, e retire o comentário do código abaixo, para
% incluir o PDF com a imagem da ficha catalográfica oficial definitiva.

%\begin{fichacatalografica}
%\includepdf{fig_ficha_catalografica.pdf}
%\end{fichacatalografica}
