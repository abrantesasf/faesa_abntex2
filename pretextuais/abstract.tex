%%%%%%%%%%%%%%%%%%%%%%%%%%%%%%%%%%%%%%%%%%%%%%%%%%%%%%%%%%%%%%%%%%%%%%%%%%%%%%%%
% abstract.tex
%
% Modelo de arquivo para uso com a classe faesaTeX2, para a formatação de
% trabalhos acadêmicos na FAESA Centro Universitário (https://www.faesa.br).
%
% Para maiores informações, visite:
%    https://github.com/abrantesasf/faesatex2
%
% Neste arquivo você deve escrever o abstract de sua monografia, ou seja, deve
% escrever o resumo em INGÊS. O resumo deve ser objetivo e listar os aspectos
% mais importantes de seu trablaho. Ao final do abstract, as keywords que você
% definiu no arquivo principal da monografia serão inseridas automaticamente.
%%%%%%%%%%%%%%%%%%%%%%%%%%%%%%%%%%%%%%%%%%%%%%%%%%%%%%%%%%%%%%%%%%%%%%%%%%%%%%%%

% Não altere as linhas a seguir:
\setlength{\absparsep}{18pt}
\begin{resumo}[Abstract]
\begin{otherlanguage*}{english}

% Começe a escrever o abstract aqui:
Lorem ipsum dolor sit amet, consectetur adipiscing elit. Praesent sollicitudin,
ligula nec dignissim tempus, velit risus malesuada eros, eu commodo metus quam
eu magna. Aenean in urna elementum, finibus tellus eget, rhoncus est. Cras at
massa et velit fermentum lacinia. Suspendisse dignissim aliquet pretium.
Maecenas volutpat pretium blandit. Sed vulputate efficitur libero, a elementum
nisi vestibulum ut. Phasellus a semper metus. Suspendisse potenti. Pellentesque
ullamcorper dui felis, vel egestas turpis tempor nec. Curabitur in lacus
faucibus, lobortis risus eget, scelerisque turpis. Cras porta sollicitudin
convallis.
 
% Não altere as linhas a seguir:
\textbf{Keywords}: \imprimirkeywords.
\end{otherlanguage*}
\end{resumo}

