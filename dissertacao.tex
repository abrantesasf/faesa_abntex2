%%%%%%%%%%%%%%%%%%%%%%%%%%%%%%%%%%%%%%%%%%%%%%%%%%%%%%%%%%%%%%%%%%%%%%%%%%%%%%%%
%%% Modelo para trabalhos monográficos (TCC, Dissertações, Teses) da FAESA
%%% Centro Universitário (https://www.faesa.br), utilizando a classe faesaTeX2.
%%%
%%% Para maiores informações, visite:
%%% https://github.com/abrantesasf/faesatex2


%%%%%%%%%%%%%%%%%%%%%%%%%%%%%%%%%%%%%%%%%%%%%%%%%%%%%%%%%%%%%%%%%%%%%%%%%%%%%%%%
%%% Classe do documento
%%% As opções padronizadas abaixo são otimizadas para a apresentação monográfica
%%% na FAESA: só altere se souber exatamente o que está fazendo.
\documentclass[
	% -- opções da classe memoir (utilizada pela abnTeX2) --
	12pt,				  % tamanho da fonte
	openright,			  % capítulos começam em página ímpar
	twoside,			  % impressão nos dois lados do papel
	a4paper,			  % tamanho do papel
	% -- opções da classe abnTeX2 (utilizada pela faesaTeX2) --
	%chapter=TITLE,		  % capítulos em letras maiúsculas
	%section=TITLE,		  % seções em letras maiúsculas
	%subsection=TITLE,	  % subseções em letras maiúsculas
	%subsubsection=TITLE, % subsubseções em letras maiúsculas
	% -- opções do pacote babel --
	english,			% idioma adicional para hifenização
	french,				% idioma adicional para hifenização
	spanish,			% idioma adicional para hifenização
	brazil				% o último idioma é o principal do documento
	]{faesatex2}


%%%%%%%%%%%%%%%%%%%%%%%%%%%%%%%%%%%%%%%%%%%%%%%%%%%%%%%%%%%%%%%%%%%%%%%%%%%%%%%%
%%% Dados da monografia
%%% Preencha conforme as instruções a seguir.

% Informe o título da monografia (lembre-se: um título objetivo e curto é melhor
% do que um título longo).
\titulo{Este é o título do TCC}

% Informe o(s) autor(es). Se houver mais de 1 autor, separar os autores com o
% comando " \and " (sem as aspas).
\autor{Primeiro Autor \and Segundo Autor}

% Informe o local no formato Cidade/UF (geralmente não é necessário alterar).
\local{Vitória/ES}

% Informe o ano.
\data{2021}

% Informe o nome do orientador (por extenso). Se for mulher, ajuste a
% qualificação que está entre colchetes.
\orientador[Orientador:]{Nome do Orientador}

% Informe o nome do coorientador (por extenso). Se for mulher, ajuste a
% qualificação que está entre colchetes. Se não houver coorientador, comentar
% a linha.
\coorientador[Coorientador:]{Nome do Coorientador}

% Informe a unidade ou o departamento de seu curso na FAESA.
\unidadefaesa{Unidade de Engenharia e Computação}

% Informe qual o seu curso na FAESA.
\cursofaesa{Graduação em Ciência da Computação}

% Informe que tipo de trabalho monográfico é este, com uma das seguintes opções:
%    Trabalho de Conclusão de Curso
%    Dissertação de Mestrado
%    Tese de Doutorado
\tipotrabalho{Trabalho de Conclusão de Curso}

% Nome oficial da FAESA (evite alterar).
\nomefaesa{FAESA Centro Universitário}

% Informe o preâmbulo da monografia (geralmente o tipo de trabalho, o objetivo,
% o nome da instituição e a área de concentração). Consulte seu orientador para
% instruções específicas se necessário, ou altere o modelo abaixo conforme
% suas necessidades (não retire os comandos "\imprimirunidadefaesa" e
% "\imprimirnomefaesa").
\preambulo{Trabalho Preliminar de Conclusão de Curso (TCC-1) apresentado à
\imprimirunidadefaesa\ da \imprimirnomefaesa, como requisito parcial para a
obtenção do grau de bacharel em Ciência da Computação.}

% Palavras-chave da monografia, em português. Essas palavras-chave serão
% incluídas automaticamente no resumo em português, e nas propriedades do
% arquivo PDF final gerado.
\palavraschave{palavra1, palavra2, palavra3, palavra4, palavra5}

% Keywords (palavras-chave) da monografia, em inglês. Essas keywords serão
% incluídas automaticamente no abstract em inglês, e nas propriedades do arquivo
% PDF final gerado.
\keywords{word1, word2, word3, word4, word5}

% Autor a ser utilizado na geração do código de Cutter da ficha catalográfica,
% no formato "Sobrenome, Nome". Atenção: se a monografia tiver dois autores,
% geralmente escreve-se aqui o nome do autor que está em primeiro lugar em
% ordem alfabética. Por exemplo: se os autores são "Fulano de Tal" e "Beltrano
% de Tal", escreveríamos aqui "Tal, Beltrano de".
\autorcutter{Sobrenome, Nome}

% Código de cutter. Gere o código com alguma ferramente online (por exemplo:
% https://www.tabelacutter.com ou https://cuttersonline.com) e escreve o código
% aqui. Atenção: como nome do autor use "Sobrenome, Nome" e como complemento ou
% marca, use a primeira letra do título da monografia, em minúscula. Se você
% tiver dificuldade, peça auxílio à bibliotecária da FAESA.
\cutter{X999x}

% Assuntos para a ficha catalográfica. É obrigatório incluir pelo menos 1
% assunto (até o total de 5). Colocar ponto após o número do assunto e após o
% próprio assunto. Comentar os que, por acaso, não forem utilizados. Se você
% tiver dificuldade, peça auxílio à bibliotecária da FAESA.
\assunto{%
1. Assunto um.
2. Assunto dois.
3. Assunto três.
4. Assunto quatro.
5. Assunto cinco.
I. Sobrenome, Nome.           % Sobrenome e nome: 2º autor OU orientador
II. Sobrenome, Nome.          % Sobrenome e nome: orientador OU coorientador
III. \imprimirnomefaesa.      % Sobrenome e nome do coorientador OU nome FAESA
IV. Título.                   % Nome FAESA OU a palavra "Título"
%V. Título.                   % A palavra "Título", SE não está na linha acima
}

% NÃO ALTERE AQUI:
%%%%%%%%%%%%%%%%%%%%%%%%%%%%%%%%%%%%%%%%%%%%%%%%%%%%%%%%%%%%%%%%%%%%%%%%%%%%%%%%
% faesa.tex
%
% Arquivo de configuração de packages para uso com a classe faesaTeX2, para a
% formatação de trabalhos acadêmicos na FAESA Centro Universitário
% (https://www.faesa.br).
%
% Para maiores informações, visite:
%    https://github.com/abrantesasf/faesatex2
%
% NÃO ALTERE NADA AQUI!
%%%%%%%%%%%%%%%%%%%%%%%%%%%%%%%%%%%%%%%%%%%%%%%%%%%%%%%%%%%%%%%%%%%%%%%%%%%%%%%%

% Imprime corretamente o nome oficial, a unidade e o curso:
\instituicao{%
  \imprimirnomefaesa
  \par
  \imprimirunidadefaesa
  \par
  \imprimircursofaesa
}




%%%%%%%%%%%%%%%%%%%%%%%%%%%%%%%%%%%%%%%%%%%%%%%%%%%%%%%%%%%%%%%%%%%%%%%%%%%%%%%%
%%% Preâmbulo com a inclusão de todos os packages necessários
%%% Não altere!
%%%%%%%%%%%%%%%%%%%%%%%%%%%%%%%%%%%%%%%%%%%%%%%%%%%%%%%%%%%%%%%%%%%%%%%%%%%%%%%%
% preambulo.tex
%
% Arquivo de chamada de todos os outros arquivos de configurações de packages
% para uso com a classe faesaTeX2, para a formatação de trabalhos acadêmicos na
% FAESA Centro Universitário (https://www.faesa.br).
%
% Para maiores informações, visite:
%    https://github.com/abrantesasf/faesatex2
%
% NÃO ALTERE NADA AQUI!
%%%%%%%%%%%%%%%%%%%%%%%%%%%%%%%%%%%%%%%%%%%%%%%%%%%%%%%%%%%%%%%%%%%%%%%%%%%%%%%%


%%%%%%%%%%%%%%%%%%%%%%%%%%%%%%%%%%%%%%%%%%%%%%%%%%%%%%%%%%%%%%%%%%%%%%%%%%%%%%%%
%%% Estruturas de controle:
%%%%%%%%%%%%%%%%%%%%%%%%%%%%%%%%%%%%%%%%%%%%%%%%%%%%%%%%%%%%%%%%%%%%%%%%%%%%%%%%
% controles.tex
%
% Arquivo de configuração de packages para uso com a classe faesaTeX2, para a
% formatação de trabalhos acadêmicos na FAESA Centro Universitário
% (https://www.faesa.br).
%
% Para maiores informações, visite:
%    https://github.com/abrantesasf/faesatex2
%
% NÃO ALTERE NADA AQUI!
%%%%%%%%%%%%%%%%%%%%%%%%%%%%%%%%%%%%%%%%%%%%%%%%%%%%%%%%%%%%%%%%%%%%%%%%%%%%%%%%

% Carrega pacotes iniciais necessários para estrutura de controle e para a
% criação e o parse de novos comandos
%\usepackage{ifthen}     % já carregado pela classe abntex2
\usepackage{xparse}
\usepackage{ifxetex}



%%%%%%%%%%%%%%%%%%%%%%%%%%%%%%%%%%%%%%%%%%%%%%%%%%%%%%%%%%%%%%%%%%%%%%%%%%%%%%%%
%%% Compilação condicional em PDF:
%%%%%%%%%%%%%%%%%%%%%%%%%%%%%%%%%%%%%%%%%%%%%%%%%%%%%%%%%%%%%%%%%%%%%%%%%%%%%%%%
% pdf.tex
%
% Arquivo de configuração de packages para uso com a classe faesaTeX2, para a
% formatação de trabalhos acadêmicos na FAESA Centro Universitário
% (https://www.faesa.br).
%
% Para maiores informações, visite:
%    https://github.com/abrantesasf/faesatex2
%
% NÃO ALTERE NADA AQUI!
%%%%%%%%%%%%%%%%%%%%%%%%%%%%%%%%%%%%%%%%%%%%%%%%%%%%%%%%%%%%%%%%%%%%%%%%%%%%%%%%

% Pacotes para compilação condicional em PDF:
\ifxetex
\else
   \usepackage{ifpdf}
\fi



%%%%%%%%%%%%%%%%%%%%%%%%%%%%%%%%%%%%%%%%%%%%%%%%%%%%%%%%%%%%%%%%%%%%%%%%%%%%%%%%
%%% Configurações de layout da página:
%%%%%%%%%%%%%%%%%%%%%%%%%%%%%%%%%%%%%%%%%%%%%%%%%%%%%%%%%%%%%%%%%%%%%%%%%%%%%%%%
% Por: Abrantes Araújo Silva Filho
%      abrantesasf@gmail.com
% URL: https://github.com/abrantesasf/faesatex2
%      https://github.com/abrantesasf/latex


%%%%%%%%%%%%%%%%%%%%%%%%%%%%%%%%%%%%%%%%%%%%%%%%%%%%%%%%%%%%%%%%%%%%%%%%%%%%%%%%
%%% Configuração do tamanho da página, margens, espaçamento entrelinhas e, se
%%% necessário, ativa a indentação dos primeiros parágrafos.
%\RequirePackage{geometry}
%\RequirePackage{setspace}
\RequirePackage{indentfirst}
\RequirePackage{microtype}
\RequirePackage{lastpage}


%%%%%%%%%%%%%%%%%%%%%%%%%%%%%%%%%%%%%%%%%%%%%%%%%%%%%%%%%%%%%%%%%%%%%%%%%%%%%%%%
%%% Configurações para autores:
%%%%%%%%%%%%%%%%%%%%%%%%%%%%%%%%%%%%%%%%%%%%%%%%%%%%%%%%%%%%%%%%%%%%%%%%%%%%%%%%
% autores.tex
%
% Arquivo de configuração de packages para uso com a classe faesaTeX2, para a
% formatação de trabalhos acadêmicos na FAESA Centro Universitário
% (https://www.faesa.br).
%
% Para maiores informações, visite:
%    https://github.com/abrantesasf/faesatex2
%
% NÃO ALTERE NADA AQUI!
%%%%%%%%%%%%%%%%%%%%%%%%%%%%%%%%%%%%%%%%%%%%%%%%%%%%%%%%%%%%%%%%%%%%%%%%%%%%%%%%

% Ajustes para os autores e afiliações, em artigos
\makeatletter
\@ifclassloaded{article}{
\usepackage{authblk}
}{}
\makeatother



%%%%%%%%%%%%%%%%%%%%%%%%%%%%%%%%%%%%%%%%%%%%%%%%%%%%%%%%%%%%%%%%%%%%%%%%%%%%%%%%
%%% Configuração para as fontes do documento:
%%% Se você quiser usar fontes próprias ou do sistema, precisará ajustar as
%%% configurações no arquivo "utils/fontes.tex".
%%% ATENÇÃO: por padrão eu utilizo XeLaTeX com fontes proprietárias projetadas
%%% por Matthew Butterick, adquiridas em https://mbtype.com, que não podem ser
%%% distribuídas devido à licença de utilização. Se você usar XeLaTeX, você
%%% DEVERÁ OBRIGATORIAMENTE ajustar as fontes no arquivo "utils/fontes.tex".
%%%%%%%%%%%%%%%%%%%%%%%%%%%%%%%%%%%%%%%%%%%%%%%%%%%%%%%%%%%%%%%%%%%%%%%%%%%%%%%%
% Por: Abrantes Araújo Silva Filho
%      abrantesasf@gmail.com
% URL: https://github.com/abrantesasf/faesatex2
%      https://github.com/abrantesasf/latex


%%%%%%%%%%%%%%%%%%%%%%%%%%%%%%%%%%%%%%%%%%%%%%%%%%%%%%%%%%%%%%%%%%%%%%%%%%%%%%%%
%%% Configurações de encoding, lingua e fontes:
\RequirePackage[T1]{fontenc}
\RequirePackage[utf8]{inputenc}
\RequirePackage{lmodern}

% Altera, se necessário, a fonte padrão do documento. Note que nem todas as
% opções abaixo funcionam perfeitamente em modo math.
%   phv = Helvetica
%   ptm = Times
%   ppl = Palatino
%   pbk = bookman
%   pag = AdobeAvantGarde
%   pnc = Adobe NewCenturySchoolBook
%\renewcommand{\familydefault}{ppl}


%%%%%%%%%%%%%%%%%%%%%%%%%%%%%%%%%%%%%%%%%%%%%%%%%%%%%%%%%%%%%%%%%%%%%%%%%%%%%%%%
%%% Sumário:
%%%%%%%%%%%%%%%%%%%%%%%%%%%%%%%%%%%%%%%%%%%%%%%%%%%%%%%%%%%%%%%%%%%%%%%%%%%%%%%%
% Por: Abrantes Araújo Silva Filho
%      abrantesasf@gmail.com
% URL: https://github.com/abrantesasf/faesatex2
%      https://github.com/abrantesasf/latex


%%%%%%%%%%%%%%%%%%%%%%%%%%%%%%%%%%%%%%%%%%%%%%%%%%%%%%%%%%%%%%%%%%%%%%%%%%%%%%%%
%%% Ajustes de sumário
\RequirePackage{tocbibind}


%%%%%%%%%%%%%%%%%%%%%%%%%%%%%%%%%%%%%%%%%%%%%%%%%%%%%%%%%%%%%%%%%%%%%%%%%%%%%%%%
%%% Matemática:
%%%%%%%%%%%%%%%%%%%%%%%%%%%%%%%%%%%%%%%%%%%%%%%%%%%%%%%%%%%%%%%%%%%%%%%%%%%%%%%%
% matematica.tex
%
% Arquivo de configuração de packages para uso com a classe faesaTeX2, para a
% formatação de trabalhos acadêmicos na FAESA Centro Universitário
% (https://www.faesa.br).
%
% Para maiores informações, visite:
%    https://github.com/abrantesasf/faesatex2
%
% Geralmente não é necessário alterar nada aqui!
%%%%%%%%%%%%%%%%%%%%%%%%%%%%%%%%%%%%%%%%%%%%%%%%%%%%%%%%%%%%%%%%%%%%%%%%%%%%%%%%

% Carrega bibliotecas e símbolos matemáticos, fontes adicionais e configura
% algumas outras opções
\usepackage{amsmath}
\usepackage{amssymb}
\usepackage{amsthm}
\usepackage{amsfonts}
\usepackage{mathrsfs}
\usepackage{proof}
\usepackage{siunitx}
  \sisetup{group-separator = {\,}}
  \sisetup{group-digits = {integer}}
  \sisetup{output-decimal-marker = {,}}
  \sisetup{separate-uncertainty}
  \sisetup{multi-part-units = single}
  \sisetup{binary-units = true}
\usepackage{bm}
\usepackage{cancel}

% Altera separador decimal via comando, se necessário (prefira o siunitx):
%\mathchardef\period=\mathcode`.
%\DeclareMathSymbol{.}{\mathord}{letters}{"3B}

\usepackage{esvect}
\usepackage{mathtools}

% Definições para teoremas, etc.
\theoremstyle{definition}
\newtheorem{definicao}{Definição}[chapter]
\newtheorem{conjecture}{Conjectura}[chapter]
\newtheorem{teorema}{Teorema}[chapter]
\newtheorem{lemma}{Lema}[chapter]
\newtheorem{corolario}{Corolário}[chapter]



%%%%%%%%%%%%%%%%%%%%%%%%%%%%%%%%%%%%%%%%%%%%%%%%%%%%%%%%%%%%%%%%%%%%%%%%%%%%%%%%
%%% Referências bibliográficas:
%%%%%%%%%%%%%%%%%%%%%%%%%%%%%%%%%%%%%%%%%%%%%%%%%%%%%%%%%%%%%%%%%%%%%%%%%%%%%%%%
% Por: Abrantes Araújo Silva Filho
%      abrantesasf@gmail.com
% URL: https://github.com/abrantesasf/faesatex2
%      https://github.com/abrantesasf/latex


%%%%%%%%%%%%%%%%%%%%%%%%%%%%%%%%%%%%%%%%%%%%%%%%%%%%%%%%%%%%%%%%%%%%%%%%%%%%%%%%
%%% Referências bibliográficas e citações
\RequirePackage[brazilian,hyperpageref]{backref}
\RequirePackage[alf]{abntex2cite}


%%%%%%%%%%%%%%%%%%%%%%%%%%%%%%%%%%%%%%%%%%%%%%%%%%%%%%%%%%%%%%%%%%%%%%%%%%%%%%%%
%%% Cores:
%%%%%%%%%%%%%%%%%%%%%%%%%%%%%%%%%%%%%%%%%%%%%%%%%%%%%%%%%%%%%%%%%%%%%%%%%%%%%%%%
% Por: Abrantes Araújo Silva Filho
%      abrantesasf@gmail.com
% URL: https://github.com/abrantesasf/faesatex2
%      https://github.com/abrantesasf/latex


%%%%%%%%%%%%%%%%%%%%%%%%%%%%%%%%%%%%%%%%%%%%%%%%%%%%%%%%%%%%%%%%%%%%%%%%%%%%%%%%
%%% Ativa suporte extendido a cores
%%% Opções de cores: usenames (16), dvipsnames (64), svgnames (150),
%%%                  x11names (300).
\RequirePackage{color}
\RequirePackage[svgnames]{xcolor}


%%%%%%%%%%%%%%%%%%%%%%%%%%%%%%%%%%%%%%%%%%%%%%%%%%%%%%%%%%%%%%%%%%%%%%%%%%%%%%%%
%%% Computação:
%%%%%%%%%%%%%%%%%%%%%%%%%%%%%%%%%%%%%%%%%%%%%%%%%%%%%%%%%%%%%%%%%%%%%%%%%%%%%%%%
% Por: Abrantes Araújo Silva Filho
%      abrantesasf@gmail.com
% URL: https://github.com/abrantesasf/faesatex2
%      https://github.com/abrantesasf/latex


%%%%%%%%%%%%%%%%%%%%%%%%%%%%%%%%%%%%%%%%%%%%%%%%%%%%%%%%%%%%%%%%%%%%%%%%%%%%%%%%
%%% Carrega packages relacionados à computação
\RequirePackage{algorithm2e}
\RequirePackage{algorithmicx}
\RequirePackage{algpseudocode}
\RequirePackage{listings}
  \lstset{literate=
    {á}{{\'a}}1 {é}{{\'e}}1 {í}{{\'i}}1 {ó}{{\'o}}1 {ú}{{\'u}}1
    {Á}{{\'A}}1 {É}{{\'E}}1 {Í}{{\'I}}1 {Ó}{{\'O}}1 {Ú}{{\'U}}1
    {à}{{\`a}}1 {è}{{\`e}}1 {ì}{{\`i}}1 {ò}{{\`o}}1 {ù}{{\`u}}1
    {À}{{\`A}}1 {È}{{\'E}}1 {Ì}{{\`I}}1 {Ò}{{\`O}}1 {Ù}{{\`U}}1
    {ä}{{\"a}}1 {ë}{{\"e}}1 {ï}{{\"i}}1 {ö}{{\"o}}1 {ü}{{\"u}}1
    {Ä}{{\"A}}1 {Ë}{{\"E}}1 {Ï}{{\"I}}1 {Ö}{{\"O}}1 {Ü}{{\"U}}1
    {â}{{\^a}}1 {ê}{{\^e}}1 {î}{{\^i}}1 {ô}{{\^o}}1 {û}{{\^u}}1
    {Â}{{\^A}}1 {Ê}{{\^E}}1 {Î}{{\^I}}1 {Ô}{{\^O}}1 {Û}{{\^U}}1
    {œ}{{\oe}}1 {Œ}{{\OE}}1 {æ}{{\ae}}1 {Æ}{{\AE}}1 {ß}{{\ss}}1
    {ű}{{\H{u}}}1 {Ű}{{\H{U}}}1 {ő}{{\H{o}}}1 {Ő}{{\H{O}}}1
    {ç}{{\c c}}1 {Ç}{{\c C}}1 {ø}{{\o}}1 {å}{{\r a}}1 {Å}{{\r A}}1
    {€}{{\euro}}1 {£}{{\pounds}}1 {«}{{\guillemotleft}}1
    {»}{{\guillemotright}}1 {ñ}{{\~n}}1 {Ñ}{{\~N}}1 {¿}{{?`}}1
  }
\definecolor{mGreen}{rgb}{0,0.6,0}
\definecolor{mGray}{rgb}{0.5,0.5,0.5}
\definecolor{mPurple}{rgb}{0.58,0,0.82}
\definecolor{backgroundColour}{rgb}{0.95,0.95,0.92}
\lstdefinestyle{CStyle}{
    backgroundcolor=\color{backgroundColour},   
    commentstyle=\color{green}\ttfamily,
    keywordstyle=\color{blue}\ttfamily,
    numberstyle=\tiny\color{mGray},
    stringstyle=\color{red}\ttfamily,
    basicstyle=\ttfamily,
    morecomment=[l][\color{magenta}]{\#},
    breakatwhitespace=false,         
    breaklines=true,                 
    captionpos=b,                    
    keepspaces=true,                 
    numbers=left,                    
    numbersep=5pt,                  
    showspaces=false,                
    showstringspaces=false,
    showtabs=false,                  
    tabsize=4,
    language=C
}


%%%%%%%%%%%%%%%%%%%%%%%%%%%%%%%%%%%%%%%%%%%%%%%%%%%%%%%%%%%%%%%%%%%%%%%%%%%%%%%%
%%% Gráficos:
%%%%%%%%%%%%%%%%%%%%%%%%%%%%%%%%%%%%%%%%%%%%%%%%%%%%%%%%%%%%%%%%%%%%%%%%%%%%%%%%
% graficos.tex
%
% Arquivo de configuração de packages para uso com a classe faesaTeX2, para a
% formatação de trabalhos acadêmicos na FAESA Centro Universitário
% (https://www.faesa.br).
%
% Para maiores informações, visite:
%    https://github.com/abrantesasf/faesatex2
%
% NÃO ALTERE NADA AQUI!
%%%%%%%%%%%%%%%%%%%%%%%%%%%%%%%%%%%%%%%%%%%%%%%%%%%%%%%%%%%%%%%%%%%%%%%%%%%%%%%%

% Suporte à importação de gráficos externos
\makeatletter
\@ifclassloaded{beamer}{
   \usepackage{graphicx}
}{
   \ifxetex
      \usepackage{graphicx}
   \else
      \ifpdf
         \usepackage[pdftex]{graphicx}
      \else
         \usepackage[dvips]{graphicx}
      \fi
   \fi
}
\makeatother


% Suporte à criação de gráficos proceduralmente no LaTeX:
\usepackage{tikz}
\usetikzlibrary{arrows,automata,backgrounds,matrix,patterns,positioning,shapes,shadows,calc}



%%%%%%%%%%%%%%%%%%%%%%%%%%%%%%%%%%%%%%%%%%%%%%%%%%%%%%%%%%%%%%%%%%%%%%%%%%%%%%%%
%%% Ícones:
%%%%%%%%%%%%%%%%%%%%%%%%%%%%%%%%%%%%%%%%%%%%%%%%%%%%%%%%%%%%%%%%%%%%%%%%%%%%%%%%
% icones.tex
%
% Arquivo de configuração de packages para uso com a classe faesaTeX2, para a
% formatação de trabalhos acadêmicos na FAESA Centro Universitário
% (https://www.faesa.br).
%
% Para maiores informações, visite:
%    https://github.com/abrantesasf/faesatex2
%
% NÃO ALTERE NADA AQUI!
%%%%%%%%%%%%%%%%%%%%%%%%%%%%%%%%%%%%%%%%%%%%%%%%%%%%%%%%%%%%%%%%%%%%%%%%%%%%%%%%

% Ícones da Creative Commons:
\usepackage{ccicons}



%%%%%%%%%%%%%%%%%%%%%%%%%%%%%%%%%%%%%%%%%%%%%%%%%%%%%%%%%%%%%%%%%%%%%%%%%%%%%%%%
%%% Blocos:
%%%%%%%%%%%%%%%%%%%%%%%%%%%%%%%%%%%%%%%%%%%%%%%%%%%%%%%%%%%%%%%%%%%%%%%%%%%%%%%%
% Por: Abrantes Araújo Silva Filho
%      abrantesasf@gmail.com
% URL: https://github.com/abrantesasf/faesatex2
%      https://github.com/abrantesasf/latex


%%%%%%%%%%%%%%%%%%%%%%%%%%%%%%%%%%%%%%%%%%%%%%%%%%%%%%%%%%%%%%%%%%%%%%%%%%%%%%%%
%%% Ambiente para posicionar boxes em posições arbitrárias
\RequirePackage{textpos}


%%%%%%%%%%%%%%%%%%%%%%%%%%%%%%%%%%%%%%%%%%%%%%%%%%%%%%%%%%%%%%%%%%%%%%%%%%%%%%%%
%%% Boxes coloridos:
%%%%%%%%%%%%%%%%%%%%%%%%%%%%%%%%%%%%%%%%%%%%%%%%%%%%%%%%%%%%%%%%%%%%%%%%%%%%%%%%
% colorbox.tex
%
% Arquivo de configuração de packages para uso com a classe faesaTeX2, para a
% formatação de trabalhos acadêmicos na FAESA Centro Universitário
% (https://www.faesa.br).
%
% Para maiores informações, visite:
%    https://github.com/abrantesasf/faesatex2
%
% NÃO ALTERE NADA AQUI!
%%%%%%%%%%%%%%%%%%%%%%%%%%%%%%%%%%%%%%%%%%%%%%%%%%%%%%%%%%%%%%%%%%%%%%%%%%%%%%%%

% Ativa suporte para boxes coloridos
\usepackage[many]{tcolorbox}



%%%%%%%%%%%%%%%%%%%%%%%%%%%%%%%%%%%%%%%%%%%%%%%%%%%%%%%%%%%%%%%%%%%%%%%%%%%%%%%%
%%% Tabelas:
%%%%%%%%%%%%%%%%%%%%%%%%%%%%%%%%%%%%%%%%%%%%%%%%%%%%%%%%%%%%%%%%%%%%%%%%%%%%%%%%
% Por: Abrantes Araújo Silva Filho
%      abrantesasf@gmail.com
% URL: https://github.com/abrantesasf/faesatex2
%      https://github.com/abrantesasf/latex


%%%%%%%%%%%%%%%%%%%%%%%%%%%%%%%%%%%%%%%%%%%%%%%%%%%%%%%%%%%%%%%%%%%%%%%%%%%%%%%%
%%% Packages para tabelas
\RequirePackage{array}
\RequirePackage{longtable}
\RequirePackage{tabularx}
\RequirePackage{tabu}
\RequirePackage{lscape}
\RequirePackage{colortbl}  
\RequirePackage{booktabs}


%%%%%%%%%%%%%%%%%%%%%%%%%%%%%%%%%%%%%%%%%%%%%%%%%%%%%%%%%%%%%%%%%%%%%%%%%%%%%%%%
%%% Ambientes de listas:
%%%%%%%%%%%%%%%%%%%%%%%%%%%%%%%%%%%%%%%%%%%%%%%%%%%%%%%%%%%%%%%%%%%%%%%%%%%%%%%%
% listas.tex
%
% Arquivo de configuração de packages para uso com a classe faesaTeX2, para a
% formatação de trabalhos acadêmicos na FAESA Centro Universitário
% (https://www.faesa.br).
%
% Para maiores informações, visite:
%    https://github.com/abrantesasf/faesatex2
%
% NÃO ALTERE NADA AQUI!
%%%%%%%%%%%%%%%%%%%%%%%%%%%%%%%%%%%%%%%%%%%%%%%%%%%%%%%%%%%%%%%%%%%%%%%%%%%%%%%%

% Packages para ambientes de listas
\makeatletter
\@ifclassloaded{beamer}{
   \usepackage{listings}
}{
   %\usepackage{enumitem}              % abntex2 já carrega
   \usepackage[ampersand]{easylist}
   %\usepackage{fancyvrb, bera}        % atualmente não necessário
}
\makeatother



%%%%%%%%%%%%%%%%%%%%%%%%%%%%%%%%%%%%%%%%%%%%%%%%%%%%%%%%%%%%%%%%%%%%%%%%%%%%%%%%
%%% Ambientes para quadros:
%%%%%%%%%%%%%%%%%%%%%%%%%%%%%%%%%%%%%%%%%%%%%%%%%%%%%%%%%%%%%%%%%%%%%%%%%%%%%%%%
% quadros.tex
%
% Arquivo de configuração de packages para uso com a classe faesaTeX2, para a
% formatação de trabalhos acadêmicos na FAESA Centro Universitário
% (https://www.faesa.br).
%
% Para maiores informações, visite:
%    https://github.com/abrantesasf/faesatex2
%
% NÃO ALTERE NADA AQUI!
%%%%%%%%%%%%%%%%%%%%%%%%%%%%%%%%%%%%%%%%%%%%%%%%%%%%%%%%%%%%%%%%%%%%%%%%%%%%%%%%

% Possibilita criação de quadros
% Ver https://github.com/abntex/abntex2/issues/176

\newcommand{\quadroname}{Quadro}
\newcommand{\listofquadrosname}{Lista de quadros}

\newfloat[chapter]{quadro}{loq}{\quadroname}
\newlistof{listofquadros}{loq}{\listofquadrosname}
\newlistentry{quadro}{loq}{0}

\setfloatadjustment{quadro}{\centering}
\counterwithout{quadro}{chapter}
\renewcommand{\cftquadroname}{\quadroname\space} 
\renewcommand*{\cftquadroaftersnum}{\hfill--\hfill}

\setfloatlocations{quadro}{hbtp}



%%%%%%%%%%%%%%%%%%%%%%%%%%%%%%%%%%%%%%%%%%%%%%%%%%%%%%%%%%%%%%%%%%%%%%%%%%%%%%%%
%%% Ambientes floats e similares:
%%%%%%%%%%%%%%%%%%%%%%%%%%%%%%%%%%%%%%%%%%%%%%%%%%%%%%%%%%%%%%%%%%%%%%%%%%%%%%%%
% floats.tex
%
% Arquivo de configuração de packages para uso com a classe faesaTeX2, para a
% formatação de trabalhos acadêmicos na FAESA Centro Universitário
% (https://www.faesa.br).
%
% Para maiores informações, visite:
%    https://github.com/abrantesasf/faesatex2
%
% NÃO ALTERE NADA AQUI!
%%%%%%%%%%%%%%%%%%%%%%%%%%%%%%%%%%%%%%%%%%%%%%%%%%%%%%%%%%%%%%%%%%%%%%%%%%%%%%%%

% Packages para suporte a ambientes floats, captions, etc.:
\usepackage{float}
\usepackage{wrapfig}
\usepackage{placeins}
\usepackage[justification=centering]{caption}
\usepackage{sidecap}
\usepackage{subcaption}



%%%%%%%%%%%%%%%%%%%%%%%%%%%%%%%%%%%%%%%%%%%%%%%%%%%%%%%%%%%%%%%%%%%%%%%%%%%%%%%%
%%% Ambientes, aliases, comandos e customizações em geral para serem
%%% utilizadas nos documentos. Defina aqui qualquer customização específica
%%% para seus documentos!
%%%%%%%%%%%%%%%%%%%%%%%%%%%%%%%%%%%%%%%%%%%%%%%%%%%%%%%%%%%%%%%%%%%%%%%%%%%%%%%%
% customizacoes.tex
%
% Arquivo de configuração de comandos para uso com a classe faesaTeX2, para a
% formatação de trabalhos acadêmicos na FAESA Centro Universitário
% (https://www.faesa.br).
%
% Para maiores informações, visite:
%    https://github.com/abrantesasf/faesatex2
%
% Crie os comandos que você precisa para usa monografia aqui!
%%%%%%%%%%%%%%%%%%%%%%%%%%%%%%%%%%%%%%%%%%%%%%%%%%%%%%%%%%%%%%%%%%%%%%%%%%%%%%%%

% Commando para ``italizar´´ palavras em inglês (e outras línguas!):
\newcommand{\ingles}[1]{\textit{#1}}

% Comando para escrever uma função e símbolos em fonte monoespaçada:
\newcommand{\funcao}[1]{\textbf{\texttt{#1}}}
\newcommand{\simbolo}[1]{\texttt{#1}}

% Produz ordinal masculino ou feminino dependendo do segundo argumento:
\newcommand{\ordinal}[2]{%
#1%
\ifthenelse{\equal{a}{#2}}%
{\textordfeminine}%
{\textordmasculine}}






%%%%%%%%%%%%%%%%%%%%%%%%%%%%%%%%%%%%%%%%%%%%%%%%%%%%%%%%%%%%%%%%%%%%%%%%%%%%%%%%
%%% Configurações para as propriedades do PDF
%%% Altere conforme as instruções a seguir:

% Esta é uma versão para web? Responder apenas com "S" ou "N"!
% Obs.: se for uma versão web, o PDF final gerado terá links visíveis e com
% cores em diversos elementos (links azuis, indicação de rodapé em vermelho,
% aquivos em magenta etc.). Se não for uma versão para web será considerada uma
% versão para impressão e o texto terá links ocultos sem cores extras, e será
% impresso totalmente em preto-e-branco (o que é apropriado para a máxima
% qualidade de impressão).
\versaoweb{S}

% NÃO ALTERE AQUI:
%%%%%%%%%%%%%%%%%%%%%%%%%%%%%%%%%%%%%%%%%%%%%%%%%%%%%%%%%%%%%%%%%%%%%%%%%%%%%%%%
% pdfconfig.tex
%
% Arquivo de configuração do PDF final gerado pela classe faesaTeX2, para a
% formatação de trabalhos acadêmicos na FAESA Centro Universitário
% (https://www.faesa.br).
%
% Para maiores informações, visite:
%    https://github.com/abrantesasf/faesatex2
%
% NÃO ALTERE NADA AQUI!
%%%%%%%%%%%%%%%%%%%%%%%%%%%%%%%%%%%%%%%%%%%%%%%%%%%%%%%%%%%%%%%%%%%%%%%%%%%%%%%%

% Configurações básicas do PDF
\makeatletter
\hypersetup{
   unicode=true,
   pdflang={pt-BR},
   bookmarksopen=false,
   bookmarksnumbered=true,
   bookmarksopenlevel=5,
   pdfdisplaydoctitle=true,
   pdfpagemode=UseOutlines,
   pdfstartview=FitH,
   pdfnewwindow=true,
   %pagebackref=true,
   %pdfcreator={TeX Live 2021: XeTeX (0.999993) + LaTeX2e (2021-06-01) + eTeX (2.6) + TeX (3.141592653)},
   %pdfproducer={xdvipdfmx (20210318)},
   pdftitle={\@title}, 
   pdfauthor={\@author},
   pdfsubject={\imprimirpreambulo},
   pdfkeywords={\imprimirpalavraschave, \imprimirkeywords}, 
   bookmarksdepth=5,
   citecolor=blue,      % cor dos links para as referências bibliográficas
   linkcolor=red,       % cor dos links internos (sumário, figuras, rodapé)
   filecolor=cyan,      % dor dos links para arquivos
   urlcolor=blue        % cor para links URL
}
\makeatother

% Ajusta os links e cores dependendo se o PDF é para divulgação web ou se é
% para impressão final em preto-e-branco. Obs.:
%   colorlinks:   false = links em caixas; true = links coloridos
%   hidelinks:    esconde os links
\ifthenelse{\equal{\imprimirversaoweb}{S}}{% Versão para web
   \hypersetup{colorlinks=true}%
}{% Versão para impressão
   \hypersetup{hidelinks, colorlinks=false}%
}




%%%%%%%%%%%%%%%%%%%%%%%%%%%%%%%%%%%%%%%%%%%%%%%%%%%%%%%%%%%%%%%%%%%%%%%%%%%%%%%%
%%% Compila o indice, se houver.
%%% Não altere!
\makeindex


%%%%%%%%%%%%%%%%%%%%%%%%%%%%%%%%%%%%%%%%%%%%%%%%%%%%%%%%%%%%%%%%%%%%%%%%%%%%%%%%
%%%%%%%%%%%%%%%%%%%%%%%%%%%%%%%%%%%%%%%%%%%%%%%%%%%%%%%%%%%%%%%%%%%%%%%%%%%%%%%%
%%%%%%%%%%%%%%%%%%%%%%%%%%%%%%%%%%%%%%%%%%%%%%%%%%%%%%%%%%%%%%%%%%%%%%%%%%%%%%%%
%%% Inicia o documento
\begin{document}

% Seleciona linguagem principal do documento (brazil, english).
\selectlanguage{brazil}

% Retira espaço extra obsoleto entre as frases (opcional).
%\frenchspacing


%%%%%%%%%%%%%%%%%%%%%%%%%%%%%%%%%%%%%%%%%%%%%%%%%%%%%%%%%%%%%%%%%%%%%%%%%%%%%%%%
%%% Elementos pré-textuais
\pretextual

% Capa: OBRIGATÓRIA.
% Não altere os comandos a seguir:
%%%%%%%%%%%%%%%%%%%%%%%%%%%%%%%%%%%%%%%%%%%%%%%%%%%%%%%%%%%%%%%%%%%%%%%%%%%%%%%%
% capa.tex
%
% Modelo de arquivo para uso com a classe faesaTeX2, para a formatação de
% trabalhos acadêmicos na FAESA Centro Universitário (https://www.faesa.br).
%
% Para maiores informações, visite:
%    https://github.com/abrantesasf/faesatex2
%
% Este arquivo cria a capa de sua monografia. NÃO ALTERE NADA AQUI!
%%%%%%%%%%%%%%%%%%%%%%%%%%%%%%%%%%%%%%%%%%%%%%%%%%%%%%%%%%%%%%%%%%%%%%%%%%%%%%%%
\pdfbookmark[0]{Capa}{capa}
\hypersetup{pageanchor=false}
\imprimircapa
\hypersetup{pageanchor=true}



% Folha de rosto: OBRIGATÓRIA.
% Não altere o comando a seguir:
%%%%%%%%%%%%%%%%%%%%%%%%%%%%%%%%%%%%%%%%%%%%%%%%%%%%%%%%%%%%%%%%%%%%%%%%%%%%%%%%
% rosto.tex
%
% Modelo de arquivo para uso com a classe faesaTeX2, para a formatação de
% trabalhos acadêmicos na FAESA Centro Universitário (https://www.faesa.br).
%
% Para maiores informações, visite:
%    https://github.com/abrantesasf/faesatex2
%
% Este arquivo cria a folha de rosto de sua monografia. NÃO ALTERE NADA AQUI!
%%%%%%%%%%%%%%%%%%%%%%%%%%%%%%%%%%%%%%%%%%%%%%%%%%%%%%%%%%%%%%%%%%%%%%%%%%%%%%%%

\imprimirfolhaderosto*



% Ficha bibliografica: OBRIGATÓRIA.
% Não altere o comando a seguir:
%%%%%%%%%%%%%%%%%%%%%%%%%%%%%%%%%%%%%%%%%%%%%%%%%%%%%%%%%%%%%%%%%%%%%%%%%%%%%%%%
% ficha.tex
%
% Modelo de arquivo para uso com a classe faesaTeX2, para a formatação de
% trabalhos acadêmicos na FAESA Centro Universitário (https://www.faesa.br).
%
% Para maiores informações, visite:
%    https://github.com/abrantesasf/faesatex2
%
% Este arquivo cria ficha catalográfica de sua monografia, automaticamente, de
% acordo com os dados que foram fornecidos pelo autor no arquivo da monografia,
% no formato padronizado seguido pela maioria das universidades e faculdades,
% por exemplo:
%    https://fichacatalografica.ufam.edu.br/ficha/create
%    https://www.rsirius.uerj.br/novo/index.php/ficha2
%    https://sistemas.ufmt.br/mfc/
%    http://fichacatalografica.ufc.br/
%    https://portal.biblioteca.ufabc.edu.br/servicos/ficha-catalografica
%    https://portal.biblioteca.ufabc.edu.br/ficha_catalografica/
%    https://www2.ufjf.br/biblioteca/ficha-catalografica/
%    http://biblioteca.unip.br/FichaCatalografica/BIBFichaCatalograficaWEB.aspx
%    https://sabi.ufrgs.br/servicos/publicoBC/ficha.php
%    https://www.tabelacutter.com/
%    https://cuttersonline.com/app/pages/home
%
% Você deve utilizar este modelo até a aprovação final do trabalho e, após isso,
% se a biblioteca da FAESA lhe fornecer uma ficha catalográfica definitiva,
% deve substituir o conteúdo deste arquivo por uma imagem da ficha final. Siga
% as instruções abaixo.
%%%%%%%%%%%%%%%%%%%%%%%%%%%%%%%%%%%%%%%%%%%%%%%%%%%%%%%%%%%%%%%%%%%%%%%%%%%%%%%%


%%%%%%%%%%%%%%%%%%%%%%%%%%%%%%%%%%%%%%%%%%%%%%%%%%%%%%%%%%%%%%%%%%%%%%%%%%%%%%%%
% Ficha catalográfica provisória:
% É gerada automaticamente, não ALTERE NADA AQUI!
\begin{fichacatalografica}
\vspace*{\fill}
\begin{center}
Dados Internacionais de Catalogação na Publicação (CIP)
\vspace{0.05cm}

\fbox{
\begin{minipage}[c][8cm]{1.5cm}
\vspace{0.4cm}
\imprimircutter
\vspace*{\fill}
\end{minipage}
\begin{minipage}[c][8cm]{12.5cm}
\vspace{0.4cm}
\imprimirautorcutter

\hspace{0.5cm} \imprimirtitulo\  / \imprimirautor. --- \imprimirdata.

\vspace{0.4cm}
\hspace{0.5cm} \thelastpage\ págs.\ :\ il.\ color.\ ;\ 30 cm.

\vspace{0.4cm}
\hspace{0.5cm} \imprimirorientadorRotulo~\imprimirorientador

\ifthenelse{\isundefined{\coorientador}}{}{
\hspace{0.5cm} \imprimircoorientadorRotulo~\imprimircoorientador
}

\hspace{0.5cm} \imprimirtipotrabalho\ (\imprimircursofaesa) ---
\imprimirnomefaesa, \imprimirunidadefaesa, \imprimirlocal, \imprimirdata.

\vspace{0.4cm}
\hspace{0.5cm} \imprimirassunto
\vspace*{\fill}
\end{minipage}
}
\vspace{-0.1cm}

Gerado automaticamente com os dados fornecidos pelo(a) autor(a).
\end{center}
\end{fichacatalografica}


%%%%%%%%%%%%%%%%%%%%%%%%%%%%%%%%%%%%%%%%%%%%%%%%%%%%%%%%%%%%%%%%%%%%%%%%%%%%%%%%
% Ficha catalográfica provisória:
% Depois que seu trabalho estiver aprovado de forma definitiva, se a biblioteca
% da FAESA lhe forneceu uma ficha catalográfica oficial e definitiva, salve
% essa filha em formato PDF. Depois, comente ou apague o código acima, da
% ficha catalográfica provisória, e retire o comentário do código abaixo, para
% incluir o PDF com a imagem da ficha catalográfica oficial definitiva.

%\begin{fichacatalografica}
%\includepdf{fig_ficha_catalografica.pdf}
%\end{fichacatalografica}


% Errata: OPCIONAL.
% Faça as modificações necessárias no arquivo "errata.tex", que está no
% diretório "pretextuais". Para incluir a errata, retire o comentário do comando
% abaixo. Se não houver errata, mantenha comentado.
%%%%%%%%%%%%%%%%%%%%%%%%%%%%%%%%%%%%%%%%%%%%%%%%%%%%%%%%%%%%%%%%%%%%%%%%%%%%%%%%%
% errata.tex
%
% Modelo de arquivo para uso com a classe faesaTeX2, para a formatação de
% trabalhos acadêmicos na FAESA Centro Universitário (https://www.faesa.br).
%
% Para maiores informações, visite:
%    https://github.com/abrantesasf/faesatex2
%
% Se você precisa incluir uma errata em sua monografia, é aqui que você fará
% isso. Converse com seu orientador sobre a pertinência de incluir uma errata.
%%%%%%%%%%%%%%%%%%%%%%%%%%%%%%%%%%%%%%%%%%%%%%%%%%%%%%%%%%%%%%%%%%%%%%%%%%%%%%%%

% Não altere a linha a seguir:
\begin{errata}

% Escreva a errata aqui:
Putz, errei tudo. Desculpe aí!

% Não altere a linha a seguir:
\end{errata}



% Folha de aprovação: OBRIGATÓRIA.
% Faça as modificações necessárias no arquivo "aprovacao.tex", que está no
% diretório "pretextuais". Não altere o comando abaixo:
%%%%%%%%%%%%%%%%%%%%%%%%%%%%%%%%%%%%%%%%%%%%%%%%%%%%%%%%%%%%%%%%%%%%%%%%%%%%%%%%
% aprovacao.tex
%
% Modelo de arquivo para uso com a classe faesaTeX2, para a formatação de
% trabalhos acadêmicos na FAESA Centro Universitário (https://www.faesa.br).
%
% Para maiores informações, visite:
%    https://github.com/abrantesasf/faesatex2
%
% Esta é a folha de aprovação (ou reprovação!, se você fizer um trabalho ruim)
% de sua monografia. Você deve utilizar este modelo até a aprovação final do
% trabalho e, após isso, deve substituir o conteúdo deste arquivo por uma
% imagem da página assinada pela banca. Siga as instruções abaixo.
%%%%%%%%%%%%%%%%%%%%%%%%%%%%%%%%%%%%%%%%%%%%%%%%%%%%%%%%%%%%%%%%%%%%%%%%%%%%%%%%


%%%%%%%%%%%%%%%%%%%%%%%%%%%%%%%%%%%%%%%%%%%%%%%%%%%%%%%%%%%%%%%%%%%%%%%%%%%%%%%%
% Folha de aprovação provisória:
% Esta folha de aprovação provisória irá imprimir, automaticamente, o nome
% do orientador, mas NÃO IMPRIMIRÁ automaticamente o nome do coorientador ou os
% nomes dos membros da banca. Para isso, você deve acrescentar o nome do
% coorientador e o nome dos membros da banca nos locais assinalados abaixo.
% NÃO ALTERE mais nada, somente onde está assinalado!

\begin{folhadeaprovacao}

  \begin{center}
    {\ABNTEXchapterfont\large\imprimirautor}

    \vspace*{\fill}\vspace*{\fill}
    \begin{center}
      \ABNTEXchapterfont\bfseries\Large\imprimirtitulo
    \end{center}
    \vspace*{\fill}
    
    \hspace{.45\textwidth}
    \begin{minipage}{.5\textwidth}
        \imprimirpreambulo
    \end{minipage}%
    \vspace*{\fill}
   \end{center}
        
   Trabalho \{a|re\}provado. \imprimirlocal, xx de xxx de 2021.

   \assinatura{\textbf{\imprimirorientador} \\ Orientador}
   % ALTERE AQUI se seu trabalho tiver coorientador:
   %\assinatura{\textbf{NOME DO COORIENTADOR} \\ Coorientador}
   % ALTERE AQUI e coloque o nome por extenso dos professores da banca:
   \assinatura{\textbf{NOME DO PROFESSOR} \\ Membro da banca}
   \assinatura{\textbf{NOME DO PROFESSOR} \\ Membro da banca}
   % ALTERE AQUI se seu trabalho tiver mais membros na banca:
   %\assinatura{\textbf{NOME DO PROFESSOR} \\ Membro da banca}
   %\assinatura{\textbf{NOME DO PROFESSOR} \\ Membro da banca}
      
   \begin{center}
    \vspace*{0.5cm}
    {\large\imprimirlocal}
    \par
    {\large\imprimirdata}
    \vspace*{1cm}
  \end{center}
  
\end{folhadeaprovacao}


%%%%%%%%%%%%%%%%%%%%%%%%%%%%%%%%%%%%%%%%%%%%%%%%%%%%%%%%%%%%%%%%%%%%%%%%%%%%%%%%
% Folha de aprovação definitiva:
% Depois que seu trabalho estiver aprovado de forma definitiva, peça para
% que seu orientador, coorientador e demais membros da banca assinem a folha
% de aprovação. Escaneia a folha de aprovação assinada, com boa resolução, e
% salve a imagem em formato PDF. Depois, comente ou apague o código acima, da
% folha de aprovação provisória, e retire o comentário do código abaixo, para
% incluir o PDF com a imagem escaneada da folha de aprovação assinada.

%\begin{folhadeaprovacao}
%\includepdf{folhadeaprovacao_final.pdf}
% \end{folhadeaprovacao}



% Dedicatória: OPCIONAL.
% Faça as modificações necessárias no arquivo "dedicatoria.tex", que está no
% diretório "pretextuais". Para incluir a dedicatória, retire o comentário do
% comando abaixo. Se não houver dedicatória, mantenha comentado.
%%%%%%%%%%%%%%%%%%%%%%%%%%%%%%%%%%%%%%%%%%%%%%%%%%%%%%%%%%%%%%%%%%%%%%%%%%%%%%%%%
% dedicatoria.tex
%
% Modelo de arquivo para uso com a classe faesaTeX2, para a formatação de
% trabalhos acadêmicos na FAESA Centro Universitário (https://www.faesa.br).
%
% Para maiores informações, visite:
%    https://github.com/abrantesasf/faesatex2
%
% Se você vai dedicar este trabalho à alguém, é aqui que você fará isso.
% Converse com seu orientador sobre a pertinência de incluir uma dedicatória em
% seu trabalho.
%%%%%%%%%%%%%%%%%%%%%%%%%%%%%%%%%%%%%%%%%%%%%%%%%%%%%%%%%%%%%%%%%%%%%%%%%%%%%%%%

% Não altere as linhas a seguir:
\begin{dedicatoria}
\vspace*{\fill}
\centering
\noindent
\textit{%

% Escreva aqui a dedicatória:
Lorem ipsum dolor sit amet, consectetur adipiscing elit. Praesent sollicitudin,
ligula nec dignissim tempus, velit risus malesuada eros, eu commodo metus quam
eu magna.

% Não altere as linhas a seguir:
}
\vspace*{\fill}
\end{dedicatoria}



% Agradecimentos: OPCIONAL.
% Faça as modificações necessárias no arquivo "agradecimentos.tex", que está no
% diretório "pretextuais". Para incluir os agradecimentos, retire o comentário
% do comando abaixo. Se não houver agradecimentos, mantenha comentado.
%%%%%%%%%%%%%%%%%%%%%%%%%%%%%%%%%%%%%%%%%%%%%%%%%%%%%%%%%%%%%%%%%%%%%%%%%%%%%%%%%
% agradecimentos.tex
%
% Modelo de arquivo para uso com a classe faesaTeX2, para a formatação de
% trabalhos acadêmicos na FAESA Centro Universitário (https://www.faesa.br).
%
% Para maiores informações, visite:
%    https://github.com/abrantesasf/faesatex2
%
% Se você tem alguém para agradecer, é aqui que você fará isso! Converse com
% seu orientador sobre a pertinência ou não de incluir uma página de
% agradecimentos.
%%%%%%%%%%%%%%%%%%%%%%%%%%%%%%%%%%%%%%%%%%%%%%%%%%%%%%%%%%%%%%%%%%%%%%%%%%%%%%%%

% Não altere a linha a seguir:
\begin{agradecimentos}

% Escreva aqui os agradecimentos:
Lorem ipsum dolor sit amet, consectetur adipiscing elit. Praesent sollicitudin,
ligula nec dignissim tempus, velit risus malesuada eros, eu commodo metus quam
eu magna. Aenean in urna elementum, finibus tellus eget, rhoncus est. Cras at
massa et velit fermentum lacinia. Suspendisse dignissim aliquet pretium.
Maecenas volutpat pretium blandit. Sed vulputate efficitur libero, a elementum
nisi vestibulum ut. Phasellus a semper metus. Suspendisse potenti. Pellentesque
ullamcorper dui felis, vel egestas turpis tempor nec. Curabitur in lacus
faucibus, lobortis risus eget, scelerisque turpis. Cras porta sollicitudin
convallis.

% Não altere a linha a seguir:
\end{agradecimentos}



% Epígrafe: OPCIONAL.
% Faça as modificações necessárias no arquivo "epigrafe.tex", que está no
% diretório "pretextuais". Para incluir a epígrafe, retire o comentário do
% comando abaixo. Se não houver epígrafe, mantenha comentado.
%%%%%%%%%%%%%%%%%%%%%%%%%%%%%%%%%%%%%%%%%%%%%%%%%%%%%%%%%%%%%%%%%%%%%%%%%%%%%%%%%
% epigrafe.tex
%
% Modelo de arquivo para uso com a classe faesaTeX2, para a formatação de
% trabalhos acadêmicos na FAESA Centro Universitário (https://www.faesa.br).
%
% Para maiores informações, visite:
%    https://github.com/abrantesasf/faesatex2
%
% Se você vai incluir uma epígrafe em seu trabalho, é aqui que você fará isso.
% Converse com seu orientador sobre a pertinência de incluir uma epígrafe.
%%%%%%%%%%%%%%%%%%%%%%%%%%%%%%%%%%%%%%%%%%%%%%%%%%%%%%%%%%%%%%%%%%%%%%%%%%%%%%%%

% Não altere as linhas a seguir:
\begin{epigrafe}
\vspace*{\fill}
\begin{flushright}
\textit{%

% Escreva a epígrafe aqui:
Lorem ipsum dolor sit amet, consectetur adipiscing elit.

% Não altere as linhas a seguir:
}
\end{flushright}
\end{epigrafe}



% Resumo em português: OBRIGATÓRIO.
% Escreva o resumo no arquivo "resumo.tex", que está no diretório "pretextuais".
% Não altere o comando abaixo.
%%%%%%%%%%%%%%%%%%%%%%%%%%%%%%%%%%%%%%%%%%%%%%%%%%%%%%%%%%%%%%%%%%%%%%%%%%%%%%%%
% resumo.tex
%
% Modelo de arquivo para uso com a classe faesaTeX2, para a formatação de
% trabalhos acadêmicos na FAESA Centro Universitário (https://www.faesa.br).
%
% Para maiores informações, visite:
%    https://github.com/abrantesasf/faesatex2
%
% Neste arquivo você deve escrever o resumo de sua monografia, ou seja, deve
% escrever o resumo em PORTUGUÊS. O resumo deve ser objetivo e listar os
% aspectos mais importantes de seu trablaho. Ao final do abstract, as
% palavras-chave que você definiu no arquivo principal da monografia serão
% inseridas automaticamente.
%%%%%%%%%%%%%%%%%%%%%%%%%%%%%%%%%%%%%%%%%%%%%%%%%%%%%%%%%%%%%%%%%%%%%%%%%%%%%%%%

% Não altere as linhas a seguir:
\setlength{\absparsep}{18pt}
\begin{resumo}

% Começe a escrever o abstract aqui:
Lorem ipsum dolor sit amet, consectetur adipiscing elit. Praesent sollicitudin,
ligula nec dignissim tempus, velit risus malesuada eros, eu commodo metus quam
eu magna. Aenean in urna elementum, finibus tellus eget, rhoncus est. Cras at
massa et velit fermentum lacinia. Suspendisse dignissim aliquet pretium.
Maecenas volutpat pretium blandit. Sed vulputate efficitur libero, a elementum
nisi vestibulum ut. Phasellus a semper metus. Suspendisse potenti. Pellentesque
ullamcorper dui felis, vel egestas turpis tempor nec. Curabitur in lacus
faucibus, lobortis risus eget, scelerisque turpis. Cras porta sollicitudin
convallis.

% Não altere as linhas a seguir:
\textbf{Palavras-chave}: \imprimirpalavraschave.
\end{resumo}



% Resumo em inglês: OBRIGATÓRIO.
% Escreva o resumo no arquivo "abstract.tex", que está no diretório
% "pretextuais". Não altere o comando abaixo.
%%%%%%%%%%%%%%%%%%%%%%%%%%%%%%%%%%%%%%%%%%%%%%%%%%%%%%%%%%%%%%%%%%%%%%%%%%%%%%%%
% abstract.tex
%
% Modelo de arquivo para uso com a classe faesaTeX2, para a formatação de
% trabalhos acadêmicos na FAESA Centro Universitário (https://www.faesa.br).
%
% Para maiores informações, visite:
%    https://github.com/abrantesasf/faesatex2
%
% Neste arquivo você deve escrever o abstract de sua monografia, ou seja, deve
% escrever o resumo em INGÊS. O resumo deve ser objetivo e listar os aspectos
% mais importantes de seu trablaho. Ao final do abstract, as keywords que você
% definiu no arquivo principal da monografia serão inseridas automaticamente.
%%%%%%%%%%%%%%%%%%%%%%%%%%%%%%%%%%%%%%%%%%%%%%%%%%%%%%%%%%%%%%%%%%%%%%%%%%%%%%%%

% Não altere as linhas a seguir:
\setlength{\absparsep}{18pt}
\begin{resumo}[Abstract]
\begin{otherlanguage*}{english}

% Começe a escrever o abstract aqui:
Lorem ipsum dolor sit amet, consectetur adipiscing elit. Praesent sollicitudin,
ligula nec dignissim tempus, velit risus malesuada eros, eu commodo metus quam
eu magna. Aenean in urna elementum, finibus tellus eget, rhoncus est. Cras at
massa et velit fermentum lacinia. Suspendisse dignissim aliquet pretium.
Maecenas volutpat pretium blandit. Sed vulputate efficitur libero, a elementum
nisi vestibulum ut. Phasellus a semper metus. Suspendisse potenti. Pellentesque
ullamcorper dui felis, vel egestas turpis tempor nec. Curabitur in lacus
faucibus, lobortis risus eget, scelerisque turpis. Cras porta sollicitudin
convallis.
 
% Não altere as linhas a seguir:
\textbf{Keywords}: \imprimirkeywords.
\end{otherlanguage*}
\end{resumo}



% Lista de ilustrações: OPCIONAL (geralmente é incluída).
% É gerada automaticamente. Para incluir, retire o comentário do comando abaixo.
%%%%%%%%%%%%%%%%%%%%%%%%%%%%%%%%%%%%%%%%%%%%%%%%%%%%%%%%%%%%%%%%%%%%%%%%%%%%%%%%
% ilustracoes.tex
%
% Modelo de arquivo para uso com a classe faesaTeX2, para a formatação de
% trabalhos acadêmicos na FAESA Centro Universitário (https://www.faesa.br).
%
% Para maiores informações, visite:
%    https://github.com/abrantesasf/faesatex2
%
% Este arquivo cria a lista de ilustrações de sua monografia.
% NÃO ALTERE NADA AQUI!
%%%%%%%%%%%%%%%%%%%%%%%%%%%%%%%%%%%%%%%%%%%%%%%%%%%%%%%%%%%%%%%%%%%%%%%%%%%%%%%%

\pdfbookmark[0]{\listfigurename}{lof}
\listoffigures*
\cleardoublepage



% Lista de quadros: OPCIONAL (geralmente NÃO é incluída).
% Para incluir, retire o comentário do comando abaixo. Tem certeza que você
% irá incluir uma lista de quadros? É realmente necessário? Tem certeza?
%%%%%%%%%%%%%%%%%%%%%%%%%%%%%%%%%%%%%%%%%%%%%%%%%%%%%%%%%%%%%%%%%%%%%%%%%%%%%%%%%
% quadros.tex
%
% Modelo de arquivo para uso com a classe faesaTeX2, para a formatação de
% trabalhos acadêmicos na FAESA Centro Universitário (https://www.faesa.br).
%
% Para maiores informações, visite:
%    https://github.com/abrantesasf/faesatex2
%
% Este arquivo cria a lista de quadros de sua monografia. Tem certeza que você
% quer incluir esse troço? De qualquer form, NÃO ALTERE NADA AQUI!
%%%%%%%%%%%%%%%%%%%%%%%%%%%%%%%%%%%%%%%%%%%%%%%%%%%%%%%%%%%%%%%%%%%%%%%%%%%%%%%%

\pdfbookmark[0]{\listofquadrosname}{loq}
\listofquadros*
\cleardoublepage



% Lista de tabelas: OPCIONAL (geralmente é incluída).
% É gerada automaticamente. Para incluir, retire o comentário do comando abaixo.
%%%%%%%%%%%%%%%%%%%%%%%%%%%%%%%%%%%%%%%%%%%%%%%%%%%%%%%%%%%%%%%%%%%%%%%%%%%%%%%%
% tabelas.tex
%
% Modelo de arquivo para uso com a classe faesaTeX2, para a formatação de
% trabalhos acadêmicos na FAESA Centro Universitário (https://www.faesa.br).
%
% Para maiores informações, visite:
%    https://github.com/abrantesasf/faesatex2
%
% Este arquivo cria a lista de tabelas de sua monografia. NÃO ALTERE NADA AQUI!
%%%%%%%%%%%%%%%%%%%%%%%%%%%%%%%%%%%%%%%%%%%%%%%%%%%%%%%%%%%%%%%%%%%%%%%%%%%%%%%%

\pdfbookmark[0]{\listtablename}{lot}
\listoftables*
\cleardoublepage



% Lista de abreviaturas e siglas: OPCIONAL (geralmente é incluída).
% Faça as alterações necessárias no arquivo "abrev.tex", que está no diretório
% "pretextuais". Para incluir, retire o comentário do comando abaixo. Se não
% houver a lista de abreviaturas e siglas, mantenha o comando comentado.
%%%%%%%%%%%%%%%%%%%%%%%%%%%%%%%%%%%%%%%%%%%%%%%%%%%%%%%%%%%%%%%%%%%%%%%%%%%%%%%%
% abrev.tex
%
% Modelo de arquivo para uso com a classe faesaTeX2, para a formatação de
% trabalhos acadêmicos na FAESA Centro Universitário (https://www.faesa.br).
%
% Para maiores informações, visite:
%    https://github.com/abrantesasf/faesatex2
%
% Este arquivo cria a lista de siglas de sua monografia (você pode incluir ou
% não uma lista de siglas). Para criar a lista de siglas, basta alterar a lista
% abaixo, incluindo as siglas apropriadas. Lembre-se de escrever as siglas
% em ORDEM ALFABÉTICA!
%%%%%%%%%%%%%%%%%%%%%%%%%%%%%%%%%%%%%%%%%%%%%%%%%%%%%%%%%%%%%%%%%%%%%%%%%%%%%%%%

% Não altere a linha a seguir:
\begin{siglas}

% Escreva aqui as siglas, no formato:
%    \item[SIGLA] Explicação da Sigla
% Inclua quantas siglas forem necessárias, em ordem alfabética!
\item[ABNT] Associação Brasileira de Normas Técnicas
\item[STEM] Science, Technology, Engineering, Mathematics

% Não altere a linha a seguir:
\end{siglas}



% Lista de símbolos: OPCIONAL (geralmente NÃO é incluída).
% Faça as alterações necessárias no arquivo "simb.tex", que está no diretório
% "pretextuais". Para incluir, retire o comentário do comando abaixo. Se não
% houver a lista de símbolos, mantenha o comando comentado.
%%%%%%%%%%%%%%%%%%%%%%%%%%%%%%%%%%%%%%%%%%%%%%%%%%%%%%%%%%%%%%%%%%%%%%%%%%%%%%%%%
% simb.tex
%
% Modelo de arquivo para uso com a classe faesaTeX2, para a formatação de
% trabalhos acadêmicos na FAESA Centro Universitário (https://www.faesa.br).
%
% Para maiores informações, visite:
%    https://github.com/abrantesasf/faesatex2
%
% Este arquivo cria a lista de símbolos de sua monografia (você pode incluir ou
% não uma lista de símbolos, converse com seu orientador para saber se será
% necessário). Para criar a lista de símbolos, basta alterar a lista
% abaixo, incluindo os símbolos apropriadas. Lembre-se de escrever os símbolos
% em uma ordem que faça sentido para o seu trabalho.
%%%%%%%%%%%%%%%%%%%%%%%%%%%%%%%%%%%%%%%%%%%%%%%%%%%%%%%%%%%%%%%%%%%%%%%%%%%%%%%%

% Não altere a linha a seguir:
\begin{simbolos}

% Escreva aqui os símbolos, no formato:
%    \item[$ simbolo-matematico $] Explicação do símbolo
% Inclua quantos símbolos forem necessários, em ordem que faça sentido.
\item[$ \Gamma $] Letra grega gama maiúscula
\item[$ \Lambda $] Letra grega lambda maiúscula
\item[$ \zeta $] Letra grega zeta minúscula
\item[$ \in $] Pertence

% Não altere a linha a seguir:
\end{simbolos}



% Sumário: OBRIGATÓRIO.
% É gerado automaticamente. Não altere o comando a seguir:
%%%%%%%%%%%%%%%%%%%%%%%%%%%%%%%%%%%%%%%%%%%%%%%%%%%%%%%%%%%%%%%%%%%%%%%%%%%%%%%%
% sumario.tex
%
% Modelo de arquivo para uso com a classe faesaTeX2, para a formatação de
% trabalhos acadêmicos na FAESA Centro Universitário (https://www.faesa.br).
%
% Para maiores informações, visite:
%    https://github.com/abrantesasf/faesatex2
%
% Este arquivo cria o sumário monografia. NÃO ALTERE NADA AQUI!
%%%%%%%%%%%%%%%%%%%%%%%%%%%%%%%%%%%%%%%%%%%%%%%%%%%%%%%%%%%%%%%%%%%%%%%%%%%%%%%%

\pdfbookmark[0]{\contentsname}{toc}
\tableofcontents*
\cleardoublepage




%%%%%%%%%%%%%%%%%%%%%%%%%%%%%%%%%%%%%%%%%%%%%%%%%%%%%%%%%%%%%%%%%%%%%%%%%%%%%%%%
%%% Elementos textuais
\textual

% OK, aqui vai TODO o conteúdo de seu trabalho! Escreva cada capítulo de sua
% monografia em um arquivo TeX separado, para facilitar a organização, e salve
% esses arquivos no diretório "textuais". O nome do arquivo deve incluir a
% extensão ".tex", por exemplo: "introducao.tex". Depois que os capítulos
% estiverem prontos, basta incluir os capítulos aqui com um comando "input"
% do seguinte modo:
%                      /input{textuais/arquivo}
%
% Note que, no comando acima, para incluir um capítulo gravado no arquivo
% "arquito.tex", não incluímos a extensão. Na FAESA os trabalhos monográficos
% geralmente seguem a seguinte estrutura:
%
% Se projeto de pesquisa:
%    Introdução (problema, hipótese, objetivos e justificativa)
%    Referencial teórico
%    Metodologia
%    Cronograma
%
% Se trabalho de conclusão de curso:
%    Introdução (problema, hipótese, objetivos e justificativa)
%    Referencial teórico
%    Metodologia
%    Resultados
%    Conclusões
%
% A estrutura acima é uma sugestão geral e não precisa ser seguida a ferro e
% fogo. Converse com seu orientador sobre a melhor forma de dividir o conteúdo
% de seu trabalho, e inclua os arquivos apropriados a seguir (na ordem em que
% devem aparecer). Obviamente a linha não deve estar comentada, não é?

%%%%%%%%%%%%%%%%%%%%%%%%%%%%%%%%%%%%%%%%%%%%%%%%%%%%%%%%%%%%%%%%%%%%%%%%%%%%%%%%%
% introducao.tex
%
% Modelo de arquivo para uso com a classe faesaTeX2, para a formatação de
% trabalhos acadêmicos na FAESA Centro Universitário (https://www.faesa.br).
%
% Para maiores informações, visite:
%    https://github.com/abrantesasf/faesatex2
%
% Este modelo mostra como inserir o capítulo de introdução de sua monografia.
% Basta informar o título e o label da introdução, e escrever o conteúdo.
%%%%%%%%%%%%%%%%%%%%%%%%%%%%%%%%%%%%%%%%%%%%%%%%%%%%%%%%%%%%%%%%%%%%%%%%%%%%%%%%

%%%%%%%%%%%%%%%%%%%%%%%%%%%%%%%%%%%%%%%%%%%%%%%%%%%%%%%%%%%%%%%%%%%%%%%%%%%%%%%%
\chapter{Introdução}
\label{sec:intro}

Escreva aqui a introdução do trabalho.



%\input{textuais/referencial}

%\input{textuais/metodologia}

%\input{textuais/resultados}

%\input{textuais/conclusoes}


%%%%%%%%%%%%%%%%%%%%%%%%%%%%%%%%%%%%%%%%%%%%%%%%%%%%%%%%%%%%%%%%%%%%%%%%%%%%%%%%
%%% Elementos pós-textuais
\postextual

% Referências bibliográficas: OBRIGATÓRIO.
% Indique, abaixo, onde está o banco de dados (geralmente um arquivo) que
% armazena todas as referências bibliográficas que você citou neste trabalho.
% Atenção: o arquivo deve estar no formato BibTeX. Para saber mais sobre o
% BibTeX, consulte:
%    http://www.bibtex.org
%    https://www.bibtex.com
% Se você não tem um software para gerenciamento de referências bibliográficas
% compatível com o BibTeX, sugerimos que você use o JabRef, que é um sistema
% de gerenciamento bibligráfico open-source com várias funcionalidades. Saiba
% mais (e faça o download) aqui:
%    https://www.jabref.org
% Onde está o arquivo com as referências bibliográficas? Você deve apontar um
% arquivo no formato BibTeX aqui (o arquivo refsdb.bib é um arquivo de exemplo
% com 2 livros cadastrados):
\bibliography{referencias/refsdb}

% Glossário: OPCIONAL (geralmente NÃO é incluído).
% Se quiser incluir um glossário, consulte a documentação do abnTeX2 para
% maiores informações!
%\glossary

% Apêndices: OPCIONAIS (você deve incluir ou não dependendo se seu trabalho tem
% ou não apêndices).
% Atenção: apêndices são diferentes de anexos! Apêndices são documentos que VOCÊ
% produziu e precisa incluir em seu trabalho para ilustrar ou documentar algo.
% Se sua monografia incluir apêndices, descomente o bloco de código abaixo e
% inclua, através de comandos "input" (conforme o exemplo) os arquivos que farão
% parte dos apêndices, na ordem que você deseja.
%\begin{apendicesenv}
%\partapendices
%%%%%%%%%%%%%%%%%%%%%%%%%%%%%%%%%%%%%%%%%%%%%%%%%%%%%%%%%%%%%%%%%%%%%%%%%%%%%%%%%
% apend1.tex
%
% Modelo de arquivo para uso com a classe faesaTeX2, para a formatação de
% trabalhos acadêmicos na FAESA Centro Universitário (https://www.faesa.br).
%
% Para maiores informações, visite:
%    https://github.com/abrantesasf/faesatex2
%
% Este modelo mostra como inserir um capítulo de apêndice em sua monografia.
% Basta informar o título e o label do apêndice, e escrever o conteúdo.
% Lembre-se que um apêndice é um documento produzido por VOCÊ. Se o documento
% foi produzido por outros autores, não é um apêndice, é um anexo!
%%%%%%%%%%%%%%%%%%%%%%%%%%%%%%%%%%%%%%%%%%%%%%%%%%%%%%%%%%%%%%%%%%%%%%%%%%%%%%%%


%%%%%%%%%%%%%%%%%%%%%%%%%%%%%%%%%%%%%%%%%%%%%%%%%%%%%%%%%%%%%%%%%%%%%%%%%%%%%%%%
\chapter{Título do apêndice}
\label{apend:titulo}

Lorem ipsum dolor sit amet, consectetur adipiscing elit. Praesent sollicitudin,
ligula nec dignissim tempus, velit risus malesuada eros, eu commodo metus quam
eu magna. Aenean in urna elementum, finibus tellus eget, rhoncus est. Cras at
massa et velit fermentum lacinia. Suspendisse dignissim aliquet pretium.
Maecenas volutpat pretium blandit. Sed vulputate efficitur libero, a elementum
nisi vestibulum ut. Phasellus a semper metus. Suspendisse potenti. Pellentesque
ullamcorper dui felis, vel egestas turpis tempor nec. Curabitur in lacus
faucibus, lobortis risus eget, scelerisque turpis. Cras porta sollicitudin
convallis.


%\end{apendicesenv}

% Anexos: OPCIONAIS (você deve incluir os não dependendo se seu trabalho tem
% ou não anexos).
% Atenção: anexos são diferentes de apêndices. Anexos são documentos de OUTROS
% autores que você precisa incluir para ilustrar ou documentar algo.
% Se sua monografia incluir anexos, descomente o bloco de código abaixo e
% inclua, através de comandos "input" (conforme o exemplo) os arquivos que farão
% parte dos anexos, na ordem que você deseja.
%\begin{anexosenv}
%\partanexos
%%%%%%%%%%%%%%%%%%%%%%%%%%%%%%%%%%%%%%%%%%%%%%%%%%%%%%%%%%%%%%%%%%%%%%%%%%%%%%%%%
% anex1.tex
%
% Modelo de arquivo para uso com a classe faesaTeX2, para a formatação de
% trabalhos acadêmicos na FAESA Centro Universitário (https://www.faesa.br).
%
% Para maiores informações, visite:
%    https://github.com/abrantesasf/faesatex2
%
% Este modelo mostra como inserir um capítulo de anexo em sua monografia. Basta
% informar o título e o label do anexo, e escrever o conteúdo. Lembre-se que
% um anexo é um documento produzido por OUTRO AUTOR que você incluirá na
% monografia. Se o documento foi produzido por você, não é um anexo, é um
% apêndice!
%%%%%%%%%%%%%%%%%%%%%%%%%%%%%%%%%%%%%%%%%%%%%%%%%%%%%%%%%%%%%%%%%%%%%%%%%%%%%%%%


%%%%%%%%%%%%%%%%%%%%%%%%%%%%%%%%%%%%%%%%%%%%%%%%%%%%%%%%%%%%%%%%%%%%%%%%%%%%%%%%
\chapter{Título do anexo}
\label{anex:titulo}

Lorem ipsum dolor sit amet, consectetur adipiscing elit. Praesent sollicitudin,
ligula nec dignissim tempus, velit risus malesuada eros, eu commodo metus quam
eu magna. Aenean in urna elementum, finibus tellus eget, rhoncus est. Cras at
massa et velit fermentum lacinia. Suspendisse dignissim aliquet pretium.
Maecenas volutpat pretium blandit. Sed vulputate efficitur libero, a elementum
nisi vestibulum ut. Phasellus a semper metus. Suspendisse potenti. Pellentesque
ullamcorper dui felis, vel egestas turpis tempor nec. Curabitur in lacus
faucibus, lobortis risus eget, scelerisque turpis. Cras porta sollicitudin
convallis.


%\end{anexosenv}

% Índice remissivo: OPCIONAL (geralmente NÃO é incluído).
% Se quiser incluir um índice remissivo, consulte a documentação do abnTeX2 para
% maiores informações!
%\phantompart
%\printindex


%%%%%%%%%%%%%%%%%%%%%%%%%%%%%%%%%%%%%%%%%%%%%%%%%%%%%%%%%%%%%%%%%%%%%%%%%%%%%%%%
%%%%%%%%%%%%%%%%%%%%%%%%%%%%%%%%%%%%%%%%%%%%%%%%%%%%%%%%%%%%%%%%%%%%%%%%%%%%%%%%
%%%%%%%%%%%%%%%%%%%%%%%%%%%%%%%%%%%%%%%%%%%%%%%%%%%%%%%%%%%%%%%%%%%%%%%%%%%%%%%%
%%% Encerra o documento
\end{document}
